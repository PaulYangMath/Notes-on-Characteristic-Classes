\documentclass[12pt]{article}
\usepackage{amsmath}
\usepackage{amsthm}
\usepackage{amsfonts}
\usepackage{amssymb}
\usepackage{latexsym}
\usepackage{tikz} 
\usepackage{esint} 
%\usepackage{epsfig}
%\usepackage{graphicx}
%\usepackage[dvips]{graphicx}




\usepackage[matrix,tips,graph,curve]{xy}

\newcommand{\mnote}[1]{${}^*$\marginpar{\footnotesize ${}^*$#1}}
\linespread{1.065}

\makeatletter

\setlength\@tempdima  {5.5in}
\addtolength\@tempdima {-\textwidth}
%\addtolength\hoffset{-0.5\@tempdima}
\addtolength\hoffset{-0.6\@tempdima}
\setlength{\textwidth}{5.5in}
\setlength{\textheight}{8.75in}
%\addtolength\voffset{-0.625in}
\addtolength\voffset{-1.2in}

\makeatother

\makeatletter 
\@addtoreset{equation}{section}
\makeatother


\renewcommand{\theequation}{\thesection.\arabic{equation}}

\theoremstyle{plain}
\newtheorem{theorem}[equation]{Theorem}
\newtheorem{corollary}[equation]{Corollary}
\newtheorem{lemma}[equation]{Lemma}
\newtheorem{proposition}[equation]{Proposition}
\newtheorem{conjecture}[equation]{Conjecture}
\newtheorem{fact}[equation]{Fact}
\newtheorem{facts}[equation]{Facts}
\newtheorem*{theoremA}{Theorem A}
\newtheorem*{theoremB}{Theorem B}
\newtheorem*{theoremC}{Theorem C}
\newtheorem*{theoremD}{Theorem D}
\newtheorem*{theoremE}{Theorem E}
\newtheorem*{theoremF}{Theorem F}
\newtheorem*{theoremG}{Theorem G}
\newtheorem*{theoremH}{Theorem H}

\theoremstyle{definition}
\newtheorem{definition}[equation]{Definition}
\newtheorem{definitions}[equation]{Definitions}
%\theoremstyle{remark}

\newtheorem{remark}[equation]{Remark}
\newtheorem{remarks}[equation]{Remarks}
\newtheorem{exercise}[equation]{Exercise}
\newtheorem{example}[equation]{Example}
\newtheorem{examples}[equation]{Examples}
\newtheorem{notation}[equation]{Notation}
\newtheorem{question}[equation]{Question}
\newtheorem{assumption}[equation]{Assumption}
\newtheorem*{claim}{Claim}
\newtheorem{answer}[equation]{Answer}
%%%%%% letters %%%%

\newcommand{\IA}{\mathbb{A}}
\newcommand{\IB}{\mathbb{B}}
\newcommand{\IC}{\mathbb{C}}
\newcommand{\ID}{\mathbb{D}}
\newcommand{\IE}{\mathbb{E}}
\newcommand{\IF}{\mathbb{F}}
\newcommand{\IG}{\mathbb{G}}
\newcommand{\IH}{\mathbb{H}}
\newcommand{\II}{\mathbb{I}}
\newcommand{\IK}{\mathbb{K}}
\newcommand{\IL}{\mathbb{L}}
\newcommand{\IM}{\mathbb{M}}
\newcommand{\IN}{\mathbb{N}}
\newcommand{\IO}{\mathbb{O}}
\newcommand{\IP}{\mathbb{P}}
\newcommand{\IQ}{\mathbb{Q}}
\newcommand{\IR}{\mathbb{R}}
\newcommand{\IS}{\mathbb{S}}
\newcommand{\IT}{\mathbb{T}}
\newcommand{\IU}{\mathbb{U}}
\newcommand{\IV}{\mathbb{V}}
\newcommand{\IW}{\mathbb{W}}
\newcommand{\IX}{\mathbb{X}}
\newcommand{\IY}{\mathbb{Y}}
\newcommand{\IZ}{\mathbb{Z}}

%%%%%%% macros %%%%%

%% my definitions %%%

\newcommand{\End}{\mathrm{End}}
\newcommand{\tr}{\mathrm{tr}}
%\newcommand{\ind}{\mathrm{ind}}

\renewcommand{\index}{\mathrm{index \,}}
\newcommand{\Hom}{\mathrm{Hom}}
\newcommand{\Aut}{\mathrm{Aut}}
\newcommand{\Trace}{\mathrm{Trace}\,}
\newcommand{\Res}{\mathrm{Res}\,}
%\newcommand{\rank}{\mathrm{rank}}
%\renewcommand{\dim}{\mathrm{dim}}

\renewcommand{\deg}{\mathrm{deg}}
\newcommand{\spin}{\rm Spin}
\newcommand{\supp}{\mathrm{supp \,}}
\newcommand{\Spin}{\rm Spin}
\newcommand{\erfc}{\rm erfc\,}
\newcommand{\sgn}{\rm sgn\,}
\newcommand{\Spec}{\rm Spec\,}
\newcommand{\diag}{\rm diag\,}
\newcommand{\Fix}{\mathrm{Fix}}
\newcommand{\Ker}{\mathrm{Ker \,}}
\newcommand{\Coker}{\mathrm{Coker \,}}
\newcommand{\Sym}{\mathrm{Sym \,}}
\newcommand{\Hess}{\mathrm{Hess \,}}
\newcommand{\grad}{\mathrm{grad \,}}
\newcommand{\Center}{\mathrm{Center}}
\newcommand{\Lie}{\mathrm{Lie}}
\newcommand{\coker}{\mathrm{coker}}

\newcommand{\ch}{\rm ch} % Chern Character

\newcommand{\rank}{\rm rank} 
%\renewcommand{\c}{\rm c}  % Chern class

\newcommand\QED{\hfill $\Box$} %{\bf QED}} 

\newcommand\Pf{\nonintend{\em Proof. }}


\newcommand\reals{{\mathbb R}} 
\newcommand\complexes{{\mathbb C}}
\renewcommand\i{\sqrt{-1}}
\renewcommand\Re{\mathrm Re}
\renewcommand\Im{\mathrm Im}
\newcommand\integers{{\mathbb Z}}
\newcommand\quaternions{{\mathbb H}}


\newcommand\iso{\,{\cong}\,} 
\newcommand\tensor{{\otimes}}
\newcommand\Tensor{{\bigotimes}} 
\newcommand\union{\bigcup} 
\newcommand\onehalf{\frac{1}{2}}
%\newcommand\Sym[1]{{Sym^{#1}(\complexes^2)}}

\newcommand\lie[1]{{\mathfrak #1}} 
\newcommand\smooth{\mathcal{C}^{\infty}}
\newcommand\trivial{{\mathbb I}}
\newcommand\widebar{\overline}

%%%%%Delimiters%%%%

\newcommand{\<}{\langle}
\renewcommand{\>}{\rangle}

%\renewcommand{\(}{\left(}
%\renewcommand{\)}{\right)}


%%%% Different kind of derivatives %%%%%

\newcommand{\delbar}{\bar{\partial}}
\newcommand{\pdu}{\frac{\partial}{\partial u}}
%\newcommand{\pd}[1][2]{\frac{\partial #1}{\partial #2}}

%%%%% Arrows %%%%%
%\renewcommand{\ra}{\rightarrow}                   % right arrow
\newcommand{\lra}{\longrightarrow}              % long right arrow
%\renewcommand{\la}{\leftarrow}                    % left arrow
\newcommand{\lla}{\longleftarrow}               % long left arrow
\newcommand{\ua}{\uparrow}                     % long up arrow
\newcommand{\na}{\nearrow}                      %  NE arrow
\newcommand{\llra}[1]{\stackrel{#1}{\lra}}      % labeled long right arrow
\newcommand{\llla}[1]{\stackrel{#1}{\lla}}      % labeled long left arrow
%\newcommand{\lua}[1]{\stackrel{#1}{\ua}}      % labeled  up arrow
%\newcommand{\lna}[1]{\stackrel{#1}{\na}}      % labeled long NE arrow
\newcommand{\xra}{\xrightarrow}
\newcommand{\into}{\hookrightarrow}
\newcommand{\tto}{\longmapsto}
\def\llra{\longleftrightarrow}

\def\d/{/\mspace{-6.0mu}/}
\newcommand{\git}[3]{#1\d/_{\mspace{-4.0mu}#2}#3}
\newcommand\zetahilb{\zeta_{{\text{Hilb}}}}
\def\Fy{\sF \mspace{-3.0mu} \cdot \mspace{-3.0mu} y}
\def\tv{\tilde{v}}
\def\tw{\tilde{w}}
\def\wt{\widetilde}
\def\wtilde{\widetilde}
\def\what{\widehat}

%%%%%%%%%%%%%%%%%%% Mark's definitions %%%%%%%%%%%%%%%%%%%%

\newcommand{\frakg}{\mbox{\frakturfont g}}
\newcommand{\frakk}{\mbox{\frakturfont k}}
\newcommand{\frakp}{\mbox{\frakturfont p}}
\newcommand{\q}{\mbox{\frakturfont q}}
\newcommand{\frakn}{\mbox{\frakturfont n}}
\newcommand{\frakv}{\mbox{\frakturfont v}}
\newcommand{\fraku}{\mbox{\frakturfont u}}
\newcommand{\frakh}{\mbox{\frakturfont h}}
\newcommand{\frakm}{\mbox{\frakturfont m}}
\newcommand{\frakt}{\mbox{\frakturfont t}}
\newcommand{\G}{\Gamma}
\newcommand{\g}{\gamma}
\newcommand{\fraka}{\mbox{\frakturfont a}}
\newcommand{\db}{\bar{\partial}}
\newcommand{\dbs}{\bar{\partial}^*}
\newcommand{\p}{\partial}
\renewcommand{\k}{\textbf{k}}
\newcommand{\rH}{\widetilde{H}}
\newcommand{\cH}{H^\ast}
\newcommand{\ccH}{\check{H}^*}
\newcommand{\ext}{\Lambda_{\IZ}}
\newcommand{\rp}{\IR\IP}
\newcommand{\hz}{\IZ/2\IZ}
\newcommand{\tf}{\wt{f}}
\newcommand{\form}{\Omega}
\newcommand{\dr}{H_{DR}}
\newcommand{\cdr}{H_{c}}
\newcommand{\plane}{\IR^2}
\newcommand{\pplane}{\IR^2 - \{0\}}
\newcommand{\fU}{\mathfrak{U}}
\newcommand{\cU}{\mathcal{U}}
\newcommand{\calH}{\mathcal{H}}
\newcommand{\cV}{\mathcal{V}}
\newcommand{\cI}{\mathcal{I}}
\newcommand{\cJ}{\mathcal{J}}
\newcommand{\cF}{\mathcal{F}}
\newcommand{\sH}{\mathcal{H}_{cv}}

%% Temporary Definitions %%
\newcommand{\noi}{dx_1 \wedge \cdots \what{dx_i} \cdots \wedge dx_n}
\newcommand{\ctsw}{dx_1 \wedge \cdots \wedge dx_n}
\newcommand{\w}{\omega}
\newcommand{\cp}{\IC\IP}
\newcommand{\cf}{f^*}
\newcommand{\Up}{U^{\pm}_i}
\newcommand{\Vp}{V^{\pm}_i}
\newcommand{\ppm}{\phi^{\pm}_i}
\newcommand{\why}{\what{y}_i}
\newcommand{\whx}{\what{x}_i}
\newcommand{\whz}{\what{z}_i}
\newcommand{\Wp}{W^{\pm}_i}
\newcommand{\wb}{\overline}
\newcommand{\cl}{\mathrm{cl}}

\newcommand{\s}{\sigma}
\newcommand{\lb}{\lambda}
\newcommand{\hint}{\int_{\IH^n}}
\newcommand{\hbint}{\int_{\p \IH^n}}
\newcommand{\sint}{\int_{-\infty}^\infty}
\newcommand{\intr}{\mathrm{int}\,}


%%%%%%%%%%%%% new definitions for the positive mass paper %%%%%%%%%

\newcommand{\sperp}{{\scriptscriptstyle \perp}}
\newcommand{\tG}{G_n(\IR^{n + k})}
\newcommand{\Ext}{\mathrm{Ext}}
\newcommand{\proj}{\mathrm{proj}}
\newcommand{\invlim}{\varprojlim}

\newcommand{\Exp}{\mathrm{Exp}}
\newcommand{\sm}{\varepsilon}
\newcommand{\Ohm}{\Omega}
\newcommand{\Sq}{\mathrm{Sq}}
%%%%%%%%%%%%%%%%%%%%%%%

%%%%%%%%%%%%%%%%%%%%%%%%%%%%%%%%%%%%%%%%%%%%%





\begin{document}


\title{Notes on Characteristic Classes}
\author{Ziquan Yang}


\date{\today}

\maketitle
 
\tableofcontents
%\setcounter{secnumdepth}{1} 

\section{Introduction}
\section{Algebraic Topology}
\subsection{A brief review of singular cohomology}

The algebraic topological view of characteristic classes is built upon singular cohomology. Therefore we brief recap some important facts about singular cohomology. Singular homology and singular cohomology are dual to each other. Indeed, cohomology groups can be computed once we know the homology groups, using \textit{universal coefficient theorem}, which says the following sequence 
$$ 0 \to \Ext(H_{n - 1}(C), G) \to H^n(C; G) \to \Hom(H_n(C), G) \to 0 $$
for a chain complex $C$ of free abelian groups. UCT is a basic fact in homological algebra and there is not much topology in it. However, singular cohomology has some advantage over homology in that there is a ring structure on $H^*(X; G)$.  \\\\
\textbf{Long exact sequence of a pair}\\
Then short exact sequence 
$$ 0 \to C_n(A) \stackrel{i}{\to} C_n(X) \to C_n(X, A) \to 0 $$
dualizes to 
$$ 0 \to C^n(X, A; G) \to C^n(X; G) \stackrel{i^*}{\to} C^n(A; G) \to 0 $$
Recall that the function $\Hom(\cdot, G)$ is always left exact, but since the above groups are all free abelian, we do have surjectivity onto $C^n(A; G)$. Indeed, given $\varphi \in C^n(A; G)$, we can extend it by $0$ on simplicies whose images are not contained in $A$. 

$i^*$ is naturally interpreted as the restriction of a map on $C_n(X)$ to a map on $C_n(A)$, and the kernel consists of those maps that are zero on $C_n(A)$, which are naturally identified with maps in $\Hom(C_n(X)/C_n(A), G)$, which is precisely $C^n(X, A; G)$. In particular, unlike $C_n(X, A; G)$, $C^n(X, A; G)$ can be naturally viewed as a subgroup of $C^n(X; G)$, i.e. those that are zero on $C(A)$, and the boundary map $\delta : C^n(X, A; G) \to C^{n+1}(X, A; G)$ is simply the restriction of $\delta : C^n(X; G) \to C^{n+1}(X; G)$. That the relative cochains form a subset of absolute cochains makes the relative cohomology conceptually easier than relative homology. \\\\
\textbf{Cross product}
Cross product is also called external cup product. The absolute version is 
$$ \times : H^k(X; R) \times H^l(Y; R) \to H^{k + l}(X \times Y; R) $$
and the relative version is 
$$ \times : H^k(X, A; R) \times H^l(Y, B; R) \to H^{k + l}(X \times Y, A \times Y \cup X \times B; R)$$
given by $a \times b = p_1^*(a) \cup p_2^*(b)$, where $p_1, p_2$ are projections from $X \times Y$ to $X$ and $Y$. \\\\
\textbf{Cohomology sequence of a triple}\\
Given a triple $A \subseteq B \subseteq X$, It is not hard to convince ourselves that the following sequence is exact:
$$ 0 \to C^\bullet (X, B; G) \to C^\bullet (X, A; G) \to C^\bullet(B, A; G) \to 0 $$
so by the snake lemma we obtain an exact sequence for a triple:
$$ \cdots \to H^k(X, B; G) \to H^k(X, A; G) \to H^k(B, A; G) \to H^{k + 1}(X, B; G) \to \cdots $$
In particular we obtain the exact sequence of pair as a special case by setting $A = \emptyset$. 

We also prepare some lemmas on commutivity of squares for future use. (cf. Hatcher p.210)

\textbf{Excision}



\subsection{Stiefel-Whitney classes}
\subsubsection{Universal bundle}
We first explain the ideas in this section. Many proofs are pretty technical topological arguments and are delayed. 

Given a curve $M^1 \subseteq \IR^{k + 1}$, we may define a map $$t : M^1 \to S^k$$ that sends every point in $M$ to a unit tangent vector. Of course, to do it consistently we need an orientation of $M^1$. However, when an orientation is not available or we may disregard it, it is always possible to define a map by passing to projective space, i.e. $t : M^1 \to \IP^k$ that sends a point to its tangent space. 

Similarly, given a hypersurface $M^k \subseteq \IR^{k + 1}$, we may define a map $$ n : M^k \to \IP^k$$ that sends a point to the orthogonal complement of its tangent space. 

More generally we would like a map for $M^n \subseteq \IR^{n + k}$, so we need to space to parametrize the $n$-dimensional subspaces of $\IR^{n + k}$. Therefore we introduce Grassmann manifolds. The underlying set $G_n(\IR^{n + k})$ is the $n$-dimensional subspaces of $\IR^{n + k}$ and we topologize it as follows: Let $$V_n(\IR^{n + k}) \subseteq \IR^{n + k} \times \cdots \times \IR^{n + k}$$ be the subset parametrizing the $n$-tuples of vectors $(v_1, \cdots, v_n)$ that are linearly independent (such tuple is called a $n$-frame). Clearly $V_n$ is a Zariski open subset. Each $n$-tuple in $V_n$ determines a $n$-dimensional subspace in $\IR^{n + k}$. Therefore there is a canonical function 
$$ q : V_n(\IR^{n + k}) \to G_n (\IR^{n + k}) $$ 
This map gives $G_n (\IR^{n + k})$ a quotient topology. Alternatively, we could have used the set of orthonomal $n$-frames $V^0_{n}$. $V^0_{n}$ is clearly compact, and hence $G_n (\IR^{n + k})$ is compact. We will verify later that $G_n (\IR^{n + k})$, which now is a priori only a topological space, is indeed a topological manifold. 

In particular $\IP^n$ can now be viewed as a special case of Grassmann manifolds, and we may define a canonical vector bundle on $G_n(\IR^{n + k})$ in a completely analogous fashion: $$\gamma^n(\IR^{n + k}) = \{ (\text{$n$-plane in }\IR^{n + k}, \text{vectors in the plane}) \}$$ 

With Grassmann mainfolds we can define generalized Gauss maps $\wb{g} : M \to \tG$ by sending a point to its tangent space. However, Grassmann manifolds can absorb much more bundles. In some sense it behaves like a terminal object in a category. 

\begin{lemma}
\label{complemma}
For any $n$-plane bundle $\xi$ over a compact base space $B$ there exists a bundle map $\xi \to \gamma^n(\IR^{n + k})$ provided that $k$ is sufficiently large. 
\end{lemma}
\begin{proof}
It suffices to define a map $\wt{f} : E(\xi) \to \IR^m$ that is linear and injective on each fiber. Since we may then define a bundle map $f : E(\xi) \to \gamma^n(\IR^{n + k})$ by 
$$ f(e) = ( \wt{f}(\text{fiber through }e), \wt{f}(e) )$$

\end{proof}

In order to generalize to common manifolds that do not happen to be compact, we do two things: replace ``compact" be ``paracompact" and introduce infinite Grassmann manifolds. A topological space is paracompact if it is Hausdorff and for every open cover, there is a locally finite refinement. Every metric space is paracompact, a direct limit of compact spaces is compact, and nearly all familiar topological spaces are paracompact. We realize the infinite Grassmann manifolds as a direct limit of the sequence 
$$ G_n(\IR^n) \subset G_n(\IR^{n+1}) \subset G_n(\IR^{n + 1}) \subset \cdots $$
To make sense of ``$\subset$": Define $\IR^\infty = \oplus_{i = 1}^\infty \IR_i$ and view each $\IR^n$ as a subset $\oplus_{i = 1}^n \IR_i$. We denote the above infinite Grassmann manifold as $G_n(\IR^\infty)$. As a special case, we have infinite projective space $\IP^\infty = G_1(\IR^\infty)$ is the direct limit of $\IP^1 \subset \IP^2 \subset \cdots$. 

A canonical bundle $\gamma^n$ over $G_n$ can be constructed just as in the finite case, i.e. as a subset of $G_n \times \IR^\infty$ that consists of pairs of planes and vectors in the planes. It is topologized as a subset of the Catesian product. We will verify that it does satisfy the local triviality condition and is a $n$-plane bundle over $G_n$. Now Lemma~\ref{complemma} can be extended to the theorem:
\begin{theorem}
Any $\IR^n$-bundle $\xi$ over a paracompact base admits a bundle map $\xi \to \gamma^n$. Moreover, any two such bundle maps are homotopic. 
\end{theorem}
In particular, the homotopic bundle maps induce homotopic maps on base spaces. Therefore any $\IR^n$-bundle $\xi$ over a paracompact space $B$ determines a unique homotopy class of maps $\wb{f_\xi} : B \to G_n$. 

\subsubsection{Grassmann manifolds}
Now we take a closer loot at the structure of Grassmann manifolds. We verify that is indeed a manifold and give a decomposition into CW complex. Finally we talk about the cohomology of Grassmann manifolds. \\\\
\textbf{A convenient open chart}\\
Recall that when working with projective space we would like to choose an affine chart and work locally. At each point $p = (x_0 : \cdots : x_n) \in \IP^n$ there is some coordinate $x_j \neq 0$, say $j = 0$. Then we can choose $A_0 = \{x_0 \neq 0\}$ to be an open neighborhood containing $p$, which is invariably easier to work with. Therefore with Grassmann manifolds $\tG$ the first thing we do is to describe how to conveniently choose an open neighborhood of a point, which in our case is some $n$ plane $X \in \IR^{n + k}$. 
Thanks to the natural inner product in $\IR^{n + k}$ we may consider the orthogonal projection map $\proj_X$ onto $X$ and let $U$ be the set of planes that project injectively onto $X$, i.e. 
$$ U = \{ X' \in \tG : X' \cap \ker \proj_X = 0 \}$$
We verify that $U$ is an open subset of $\tG$. Since the topology on $\tG$ is the quotient topology from $V_n(\IR^{n + k})$, it is equivalent to verifying that the preimage of $U$ in $V_n(\IR^{n + k})$ is open. Choose a basis $w_1, \cdots, w_k$ of $X^\perp = \ker \proj_X = 0$, a $n$-frame $(v_1, \cdots, v_n) \in V_n(\IR^{n + k})$ has a span disjoint from $X^\perp$ if and only if $\{v_1, \cdots, v_n, w_1, \cdots, w_k\}$ is a basis of $\IR^{n + k }$. Therefore we define a map $\det : V_n(\IR^{n + k}) \to \IR $ by sending $(v_1, \cdots, v_n)$ to $\det([v_1; \cdots; v_n; w_1; \cdots; w_k])$. The map is clearly continuous and $U$ is evidently $\det^{-1}(\IR - \{0\})$ and hence is open. Just as those standard affine charts on $\IP^n$, $U$ is also Zariski dense in $\tG$. 

We can view planes in $U$ as the graph of some linear map $X \to X^\perp$, so there is a natural bijective correspondence between $U$ and $\Hom(X, X^\perp) \iso M^{n \times k} = \IR^{nk}$. Indeed, we verify that it is a homeomorphism. Actually this is clear once we write down such a correspondence $T : U \to M^{n \times k}$ explicitly using coordinates. Let $v_1, \cdots, v_n$ be an orthonormal basis for $X$. For each $X' \in U$, there is a unique basis $\{w_1, \cdots, w_n\}$ such that $\proj_X w_j = v_j$ for all $j$. Then we set $$T(X') = [ \proj_{X^\perp} w_1, \cdots, \proj_{X^\perp} w_n ]$$ i.e. the matrix whose columns are $\proj_{X^\perp} w_j$. $w_j$'s evidently depends continuously on $X'$ and vice versa, so $T$ is indeed a homeomorphism. In particular, it implies that $\tG$ is Hausdorff, which could easily shown independently as well. \\\\
\textbf{Local trivialization of $\gamma^n(\IR^{n+k})$}\\
With $U$ it is also not hard to show that $\gamma^n(\IR^{n + k})$ is a bundle. Of course, the only interesting thing to verify is local triviality. On $U$, $\gamma^n(\IR^{n + k})$ restricts to pairs
$$ \{(X', v) : X' \cap \ker \proj_X = 0, v \in X'\}$$
Therefore we may use $h: \pi^{-1}(U) \to U \times X$ defined by $h(X', v) = (X', \proj_X v)$ as a local trivialization. That $X' \in U$ guarantees that $v \to \proj_X v$ is injective. It is not hard to verify that it is indeed a homeomorphism. 

In this section we use $D^p$ to denote the closed unit disk in $\IR^p$. 
Let us recall the definition of a CW-complex:
\begin{definition} A CW-complex is a Hausdorff space $K$, together with a partitiion of $K$ into disjoint subsets $\{ e_\alpha \}$, such that 
\begin{enumerate}
\item Each $e_\alpha$ is topologically an open cell of dimension $n(e)$. For each $e_\alpha$, there is a characteristic map $\varphi_\alpha : D^{n(\alpha)} \to K$ that carries $\intr D^{n(\alpha)}$ homeomorphically onto $e_\alpha$. 
\item Each point $x \in \cl(e_\alpha) - e_\alpha$ must lie in a cell $e_\beta$ of lower dimension. 
\item Each point of $K$ is contained in a finite subcomplex. 
\item $K$ is topologized as a direct limit of its finite subcomplexes. 
\end{enumerate}
\end{definition}
The last two conditions always hold if the complex is finite (i.e. has finitely many cells), so we only use them to deal with infinite complexes. 
Note that the closure of a cell need not be a cell. For example, $S^n$ can be considered a cell conplex with with a $n$-cell and a $0$-cell, but the closure of the $n$-cell is the entire sphere. Every CW-complex is paracompact. 
We first study the cell structure for $G_n(\IR^m)$. Any $n$-plane $X \subseteq \IR^m$ is filtrated by 
$$0 \subseteq X \cap \IR^1 \subseteq X \cap \IR^2 \subseteq \cdots \subseteq X \cap \IR^m $$ 
The dimension of spaces in the above filtration increases from $0$ and stabilizes at $n$. There are precisely $n$ jumps. We use a Schubert symbol $\sigma = ( \sigma_1, \cdots, \sigma_n)$ to record a jump type, i.e. $$ \dim X \cap \IR^{\sigma_i} = i, \, \dim X \cap \IR^{\sigma_{i-1}} = i - 1$$
For each Schubert symbol $\sigma$, we use $e(\sigma)$ to denote the set of all $n$-planes with the same jump type. We will show that $e(\sigma)$ is an open cell of dimension $d(\sigma) = (\sigma_1 - 1) + (\sigma_2 - 2) + \cdots + (\sigma_n - n)$. 

Let us use the standard basis for $\IR^n$ and $\IR^\infty$. For each $n$-plane $X \subseteq \IR^m$ we can write down a $n \times m$ matrix whose row space is $X$, i.e. a generator matrix for $X$. Then we reduce the matrix to reduced echelon form, then the columns with pivotal entries are in the support of $\sigma$. With some linear algebra, it is not hard to show that each $n$-plane $X \in e(\sigma)$ possesses a unique othornormal basis $(x_1, \cdots, x_n) \in H^{\sigma_1} \times \cdots \times H^{\sigma_n}$. 

Define $$\wt{e}(\sigma) = V_n^0 \cap (H^{\sigma_1} \times \cdots \times H^{\sigma_n})$$ and $$ \wb{e}(\sigma) = V_n^0 \cap (\wb{H}^{\sigma_1} \times \cdots \times \wb{H}^{\sigma_n})$$

\begin{lemma}
\end{lemma}

\subsubsection{Existence of Stiefel-Whitney classes}
This section assumes Thom isomorphism. Recall that the Thom isomorphism $\phi : H^k(B) \to H^{k + n} (E, E_0)$ can be see as the composition: 
$$ H^k(B) \stackrel{\pi^*}{\longrightarrow} H^k(E) \stackrel{\cup u}{\longrightarrow} H^{k + n}(E, E_0)$$ where $u \in H^n(E, E_0)$ is the fundamental cohomology class. 

Now we introduce Steenrod squaring operations. In general, a cohomology operation is a \textit{natural transformation} betwenn cohomology functors, they do not need to be homomorphisms of graded rings. Steenrod squaring operations are characterized by four properties:
\begin{enumerate}
\item For each pair $Y \subseteq X$, integers $n, i \ge 0$ there is an additive homomorphism:
$$ \Sq^i : H^n(X, Y) \to H^{n + i}(X, Y) $$
\item \textit{Naturality} : If $f : (X, Y) \to (X', Y')$, then $\Sq^i \circ f^*  = f^* \circ \Sq^i$. Of course, this should be satisfied by all cohomology operations. 
\item If $a \in H^n(X, Y)$, then $\Sq^0(a) = a$, $\Sq^n(a) = a \cup a$ and $\Sq^i(a) = 0$ for all $i > n$. Since higher squaring operations vanish, we define the total squaring operation as 
$$ \Sq(a) = a + \Sq^1(a) + \cdots + \Sq^n(a) $$
\item \textit{Cartan formula}, which says that cup products correspond to convolution:
$$ \Sq^k (a \cup b) = \sum_{i + j = k} \Sq^i(a) \cup \Sq^j(b) $$ Or equivalently, 
$$ \Sq(a \cup b) = \Sq(a) \cup \Sq(b) $$ Note that with naturality it is not hard to show that cross product also satisfies Cartan formula: 
$$ \Sq(a \times b) = \Sq(a) \times \Sq(b) $$
\end{enumerate}
We will construct Steenrod squaring operations later. Now we may finally define the Stiefel-Whitney classes $w_i(\xi) \in H^i(B)$ using the property:
$$ \phi(w_i(\xi)) = \Sq^i(u) $$




\subsection{Orientation and Euler classes}
An orientation of a vector space is a choice of an equivalence class of basis. Two basis are deemed as equivalent if the change of basis matrix has positive determinant, so there are exactly two orientations. 

The first occasion in which I met the concept of orientation is in a differential geometric setting. An orientation means a consistent choice of basis of $T_p M$. A manifold is orientable if there exists a chart of open sets such that the transition functions on the overlaps have positive Jacobian determinants everywhere. From this point of view it does not take too much effort to show that $M$ is orientable if and only if its tangent bundle $TM$ is orientable. 

There is also an algebraic topological description of orientation, using the concept of local cohomology/homology. Again we first deal with the orientation of a vector space. An orientation of $\IR^n$ is a choice of generator of $$H^n(\IR^n, \IR^n_0) \iso H^{n-1}(\IR^n_0) \iso H^{n - 1}(S^{n - 1}) = \IZ$$
In order to connect to the previous definition, we give a rule to associate a homology class with a choice of basis. Take $\IR^3$ as an example, we place the barycenter of a tetrahedron as the origin. Suppose the vertices of the tetrahedron (which we should think of as a representative of a cocyle in $H^n(\IR^n_0)$), are labeled by $v_0, v_1, v_2, v_3$, then we take $v_1 - v_0, v_2 - v_1, v_3 - v_2$ as a basis for $\IR^3$. The case for other $\IR^n$ is entirely similar. 

On a manifold $M$, we define the local orientation at a point $x_0 \in M$ as a choice of generator of $H^n(M, M-\{x_0\})$. Choose $U \iso \IR^n \subseteq M$ to be a neighborhood of $x_0$, by excision we have $H^n(M, M-\{x_0\}) = H^n(U, U-\{x_0\})$. If $\cl{V} \subseteq U$ is a smaller neighborhood, then by excision again the natural restriction $H^n(U, U-\{x_0\}) \to H^n(V, V-\{x_0\})$ is an isomorphism, so a generator restricts naturally to a generator. Now an orientation of $M$ is a consistent choice of local orientations such that if $x, y$ lie in a same neighborhood $\IR^n$, such that there is an open ball of finite radius $B$ containing both $x$ and $y$ and a generator of $H^n(\IR^n, \IR^n - B)$ restricts to the chosen generators at $x$ and $y$. 




\subsection{Thom Isomorphism}
\subsubsection{With $\IZ/2\IZ$ coefficients}
For the first half of the section, $\IZ/2\IZ$ coefficients are assumed.
For $n = 1$, the cohomology group $H^1(\IR, \IR_0) \iso \IZ/2\IZ$. Of course, this can be shown by abstract means, but in order to see what the generator $e^1$ looks like I decide to do it by hand. Recall that $$C^1(\IR, \IR_0)= \{ \varphi : \{ [a, b]: a, b \in \IR \} \to \IZ/2\IZ : \forall [a, b] \subseteq \IR_0, \varphi([a, b]) = 0 \}$$ If in addition $\varphi$ is a cocycle, then $\varphi([0, 0]) = 0$. In fact $\varphi$ is fixed once $\varphi([0, 1])$ is fixed. If $\varphi([0, 1]) = 0$, then $\varphi$ is trivial. If $\varphi([0, 1]) = 1$, then $$ \varphi([a, b]) = \begin{cases} 1, &\text{ if }ab = 0, a \neq b \\
 0,& \text{otherwise} \end{cases} $$ and in this case $e^1 = [\varphi]$. Technically we should have used $\sigma : [0,1] \to \IR$ instead of $[a, b]$. The correspondence is $a = \sigma(0), b = \sigma(1)$. There are many different continuous mappings with same endpoints, but in the end we may disregard this difference. 

When $n > 1$, then consider the long exact sequence of the pair $(\IR^n, \IR^n_0)$. Since $H^{n - 1}(\IR^n) = H^n(\IR^n) = 0$, we have that $$H^n(\IR^n, \IR^n_0) = H^{n - 1}(\IR^n_0) = \IZ / 2\IZ$$ 
We set $e^n = e^1 \times \cdots e^1 \in H^n(\IR^n, \IR^n_0)$. It is not hard to check, using the explicit description of $e^1$, that $e^n \neq 0$. Taking $n = 2$ for example, we see that by the definition of cross product, then following simplex won't be mapped to $0$ by $e^2$. 

The proof of this version Thom isomorphism is rather analogous to that on differential forms that I have seen before. We deal with the trivial bundle first, and then use Mayer-Vietoris + five lemma to glue things up.  

I feel the following assumes the role of Poincare lemma, but what exactly is the analogy of relative cohomology? We did not use relative cohomology that much in the differential geometric setting. 
\begin{lemma}
For any topological space $B$ and $n \ge 1$, the map $\times e^n : H^j(B) \to H^{n + j}(B \times \IR^n, B \times \IR^n_0)$ is an isomorphism. 
\end{lemma}
Note that the lemma says much more than that $H^j(B) \iso H^{n + j}(E, E_0)$, but there is an explicit one given by $\times e^n$. This is important since we may then use some of the commutative diagrams proved before. 
\begin{proof}
(commutative diagram to be found in page 267)
\end{proof}

\begin{theorem}
\label{ThomIsoMod2}
There is a unique class $u \in H^n(E, E_0)$ whose restriction to each $H^n(F, F_0)$ is nonzero for each fiber $F$. Furthermore the map $\cup u : H^j(E) \to H^{j + n}(E, E_0)$ is an isomorphism for each integer $j$. 
\end{theorem}
\begin{proof}
\textbf{trivial}\\
We first deal with the case that the bundle is trivial, i.e. $E \iso B \times \IR^n$. By the above lemma we have $$H^0(B) \stackrel{\times e^n}{\longrightarrow} H^{n}(B \times \IR^n, B \times \IR^n_0) = H^n(E, E_0)$$
Let $b_0 \in B$ be a point and let $F$ be the fiber over it, we have a commutative diagram: 

Therefore that if $s \in H^0(B)$ corresponds to $u$ under the isomorphism $\times e^n$, we have that $s$ is nonzero on each point of $B$. Of course, there is only one such class (this is actually also true on the level of cocycles), i.e. $1 \in H^0(B)$. Any element in $H^j (B \times \IR^n)$ can be written uniquely as $y \times 1$ with $y \in H^j(B)$. Now we check that 
$$ y \times 1 \mapsto (y \times 1)\cup (1 \times e) = y \times e$$
is indeed an isomorphism $H^j(E) \to H^{j + n}(E, E_0)$ by the previous lemma. \\\\
\textbf{compact}\\
Now we assume $B$ is compact, so that $B$ can be covered by finitely many open sets $U_1, \cdots, U_2$ such that $E(U_i)$ is trivial for each $i$. Since we have shown that the conclusion holds for each $E(U_i)$, it suffices to show that in general if the conclusion holds for $E(A), E(B)$, where $A, B \subseteq B$ are open sets, then it holds for $E(A \cup B)$ as well. To simply notation, label $E(A) = E', E(B) = E'', E(A\cup B ) = E, E(A \cap B) = E^\cap$. There is a relative Mayer-Vietoris sequence:
$$ \cdots \to H^n(E, E_0) \stackrel{\Psi}{\to} H^n(E', E'_0) \oplus H^n(E'', E''_0) \stackrel{\Phi}{\to} H^n(E^\cap, E^\cap_0) \to \cdots $$ 
Let $u' \in H^n(E', E'_0), u'' \in H^n(E'', E''_0)$ be the unique classes as in the theorem. Then $u'|_{E^\cap}, u''|_{E^\cap}$ both work as the unique class in $H^n(E^\cap, E^\cap_0)$ as in the theorem since the condition is local. In particular, they are equal. Therefore $(u, v) \in \ker \Phi$, and by exactness of the sequence there exists $u \in H^n(E, E_0)$ that restricts to $u'$ on $A $ and to $u''$ on $B$. This $u$ is unique, since $H^{n - 1}(E^\cap, E^\cap_0) = 0$.

To verify the second statement, consider the exact sequence: 
$$ \cdots \to H^j(E) \to H^j(E') \oplus H^j(E'') \to H^j(E^\cap) \to \cdots $$
Now apply $\cup u $ to each column, the above sequence maps to 
$$ \cdots \to H^{n + j}(E, E_0) \stackrel{\Psi}{\to} H^{n + j}(E', E'_0) \oplus H^{n + j}(E'', E''_0) \stackrel{\Phi}{\to} H^{n + j}(E^\cap, E^\cap_0) \to \cdots $$ 
$\cup u$ induces an isomorphism on the second and third column, and therefore also induces an isomorphism on the first. \\\\
\textbf{limit}
Now for a general topological space $B$, we have verified the conclusion for all compact subsets $C \subseteq B$. $B$ can be seen as the direct limit of all its compact subspaces. For each $C$, the inclusion $\iota : C \to B$ induces a map $H^j(B) \to H^j(C)$. The proof is complete by the following lemma. 
\end{proof}
\begin{lemma}
The natural homomorphism $$ H^j(B) \to \invlim H^j(C) $$ is an isomorphism. 
\end{lemma}
\begin{proof}
The corresponding statement for homology, i.e. $\varinjlim H_j(C) = H_j(B)$ is true for arbitrary coefficients. (Remember to come back to check this) Now apply UCT. Thanks to that fact that we are using field coefficients, there are no torsion groups and $H^j(\cdot) = \Hom(H_j(\cdot), \IZ/2\IZ)$.
\end{proof}

\subsection{Computations in a smooth manifold}
Let $M$ be a manifold of dimension $m$ smoothly embedded in a Riemannian manifold $A$ of dimension $m + k$. Let $\nu_k$ be the normal bundle of $M$ in $A$ and let $E$ be its total bundle. 
We first introduce the tubular neighborhood lemma. 

\begin{lemma}
There exists an open neighborhood of $M$ in $A$ that is diffeomorphic to the total bundle $E$. 
\end{lemma}
\begin{proof}
We only show the case when $M$ is compact. The lemma is also true for noncompact manifolds, but the proof is more technical. Let $(x, 0) \in E$ be a point. There is a neighborhood $U_x \subseteq E$ of $(x, 0)$ that is diffeomorphic to an open subset of $A$ containing $x$ under the exponential map. Recall how exponential map is defined: let $(x', v) \in U$ be a point, $\Exp(x', v) = \gamma(1)$ where $\gamma : [0, 1] \to A$ is the unique geodesic with $\gamma(0) = x', \gamma'(0) = v$ ($E \subseteq \tau_A$ so $\gamma'(0) = v$ makes sense). Since $M$ is compact, we can cover it with finitely many such neighborhoods $U_{x_1}, \cdots, U_{x_p}$ , and there exists a $\sm >0 $, such that 
$$ E(\sm) : = \{(x, v) \in E : \| v\| < \sm \} \subseteq \union_{i = 1}^p U_{x_i}$$ 
The above construction ensures that each $E(\sm) \cap U_{x_i}$ is diffeomorphic to its image. Now we show we can restrict $\sm$ to ensure that $E(\sm) \to A$ is injective. Suppose not, there for each $k$, we can find $(x_k, v_k), (x'_k, v'_k) \in E(1/k)$ such that $(x_k, v_k) \neq (x'_k, v'_k)$, but $\Exp(x_k, v_k) = \Exp(x'_k, v'_k)$. Since $M$ is compact, both sequences converge, say $(x_k, v_k) \to (x, 0)$ and $(x'_k, v'_k) \to (x', 0)$. Since $\Exp$ is in particular continuous, we have that $\Exp(x, 0) = \Exp(x', 0)$, but it contradicts the fact that $\Exp$ is a diffeomorphism near $\Exp(x, 0)$. It is not hard to see that $E(\sm) \iso E$. 
\end{proof}
The denote the tubular neighborhood (the image of $E(\sm)$) constructed above by $N_\sm$. By excision, the inclusion $\iota : (N_\sm, N_\sm - M) \to (A, A - M)$ induces an isomorphism $H^*(A, A - M; R) \to H^*(N_\sm, N_\sm - M; R)$. Now compose with the isomorphism induced by the diffeomorphism $E \iso N_\sm$ we have that 
$$ H^*(E, E_0; R) \iso H^*(A, A - M; R) $$
In addition, the isomorphism is canonical. 


\section{Differential Geometry}

\section{Algebraic Geometry}

\end{document}
