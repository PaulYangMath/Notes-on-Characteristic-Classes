\documentclass[12pt]{article}
\usepackage{amsmath}
\usepackage{amsthm}
\usepackage{amsfonts}
\usepackage{amssymb}
\usepackage{latexsym}
\usepackage{tikz} 
\usepackage{esint} 
%\usepackage{epsfig}
%\usepackage{graphicx}
%\usepackage[dvips]{graphicx}
\usepackage{tikz}
\usepackage{tikz-cd}
\usetikzlibrary{calc}


\usepackage[matrix,tips,graph,curve]{xy}

\newcommand{\mnote}[1]{${}^*$\marginpar{\footnotesize ${}^*$#1}}
\linespread{1.065}

\makeatletter

\setlength\@tempdima  {5.5in}
\addtolength\@tempdima {-\textwidth}
%\addtolength\hoffset{-0.5\@tempdima}
\addtolength\hoffset{-0.6\@tempdima}
\setlength{\textwidth}{5.5in}
\setlength{\textheight}{8.75in}
%\addtolength\voffset{-0.625in}
\addtolength\voffset{-1.2in}

\makeatother

\makeatletter 
\@addtoreset{equation}{section}
\makeatother


\renewcommand{\theequation}{\thesection.\arabic{equation}}

\theoremstyle{plain}
\newtheorem{theorem}[equation]{Theorem}
\newtheorem{corollary}[equation]{Corollary}
\newtheorem{lemma}[equation]{Lemma}
\newtheorem{proposition}[equation]{Proposition}
\newtheorem{conjecture}[equation]{Conjecture}
\newtheorem{fact}[equation]{Fact}
\newtheorem{facts}[equation]{Facts}
\newtheorem*{theoremA}{Theorem A}
\newtheorem*{theoremB}{Theorem B}
\newtheorem*{theoremC}{Theorem C}
\newtheorem*{theoremD}{Theorem D}
\newtheorem*{theoremE}{Theorem E}
\newtheorem*{theoremF}{Theorem F}
\newtheorem*{theoremG}{Theorem G}
\newtheorem*{theoremH}{Theorem H}

\theoremstyle{definition}
\newtheorem{definition}[equation]{Definition}
\newtheorem{definitions}[equation]{Definitions}
%\theoremstyle{remark}

\newtheorem{remark}[equation]{Remark}
\newtheorem{remarks}[equation]{Remarks}
\newtheorem{exercise}[equation]{Exercise}
\newtheorem{example}[equation]{Example}
\newtheorem{examples}[equation]{Examples}
\newtheorem{notation}[equation]{Notation}
\newtheorem{question}[equation]{Question}
\newtheorem{assumption}[equation]{Assumption}
\newtheorem*{claim}{Claim}
\newtheorem{answer}[equation]{Answer}
%%%%%% letters %%%%

\newcommand{\IA}{\mathbb{A}}
\newcommand{\IB}{\mathbb{B}}
\newcommand{\IC}{\mathbb{C}}
\newcommand{\ID}{\mathbb{D}}
\newcommand{\IE}{\mathbb{E}}
\newcommand{\IF}{\mathbb{F}}
\newcommand{\IG}{\mathbb{G}}
\newcommand{\IH}{\mathbb{H}}
\newcommand{\II}{\mathbb{I}}
\newcommand{\IK}{\mathbb{K}}
\newcommand{\IL}{\mathbb{L}}
\newcommand{\IM}{\mathbb{M}}
\newcommand{\IN}{\mathbb{N}}
\newcommand{\IO}{\mathbb{O}}
\newcommand{\IP}{\mathbb{P}}
\newcommand{\IQ}{\mathbb{Q}}
\newcommand{\IR}{\mathbb{R}}
\newcommand{\IS}{\mathbb{S}}
\newcommand{\IT}{\mathbb{T}}
\newcommand{\IU}{\mathbb{U}}
\newcommand{\IV}{\mathbb{V}}
\newcommand{\IW}{\mathbb{W}}
\newcommand{\IX}{\mathbb{X}}
\newcommand{\IY}{\mathbb{Y}}
\newcommand{\IZ}{\mathbb{Z}}

%%%%%%% macros %%%%%

%% my definitions %%%

\newcommand{\End}{\mathrm{End}}
\newcommand{\tr}{\mathrm{tr}}
%\newcommand{\ind}{\mathrm{ind}}

\renewcommand{\index}{\mathrm{index \,}}
\newcommand{\Hom}{\mathrm{Hom}}
\newcommand{\Aut}{\mathrm{Aut}}
\newcommand{\Trace}{\mathrm{Trace}\,}
\newcommand{\Res}{\mathrm{Res}\,}
%\newcommand{\rank}{\mathrm{rank}}
%\renewcommand{\dim}{\mathrm{dim}}

\renewcommand{\deg}{\mathrm{deg}}
\newcommand{\spin}{\rm Spin}
\newcommand{\supp}{\mathrm{supp \,}}
\newcommand{\Spin}{\rm Spin}
\newcommand{\erfc}{\rm erfc\,}
\newcommand{\sgn}{\rm sgn\,}
\newcommand{\Spec}{\rm Spec\,}
\newcommand{\diag}{\rm diag\,}
\newcommand{\Fix}{\mathrm{Fix}}
\newcommand{\Ker}{\mathrm{Ker \,}}
\newcommand{\Coker}{\mathrm{Coker \,}}
\newcommand{\Sym}{\mathrm{Sym \,}}
\newcommand{\Hess}{\mathrm{Hess \,}}
\newcommand{\grad}{\mathrm{grad \,}}
\newcommand{\Center}{\mathrm{Center}}
\newcommand{\Lie}{\mathrm{Lie}}
\newcommand{\coker}{\mathrm{coker}}

\newcommand{\ch}{\rm ch} % Chern Character

\newcommand{\rank}{\rm rank} 
%\renewcommand{\c}{\rm c}  % Chern class

\newcommand\QED{\hfill $\Box$} %{\bf QED}} 

\newcommand\Pf{\nonintend{\em Proof. }}


\newcommand\reals{{\mathbb R}} 
\newcommand\complexes{{\mathbb C}}
\renewcommand\i{\sqrt{-1}}
\renewcommand\Re{\mathrm Re}
\renewcommand\Im{\mathrm Im}
\newcommand\integers{{\mathbb Z}}
\newcommand\quaternions{{\mathbb H}}


\newcommand\iso{{\cong}} 
\newcommand\tensor{{\otimes}}
\newcommand\Tensor{{\bigotimes}} 
\newcommand\union{\bigcup} 
\newcommand\onehalf{\frac{1}{2}}
%\newcommand\Sym[1]{{Sym^{#1}(\complexes^2)}}

\newcommand\lie[1]{{\mathfrak #1}} 
\newcommand\smooth{\mathcal{C}^{\infty}}
\newcommand\trivial{{\mathbb I}}
\newcommand\widebar{\overline}

%%%%%Delimiters%%%%

\newcommand{\<}{\langle}
\renewcommand{\>}{\rangle}

%\renewcommand{\(}{\left(}
%\renewcommand{\)}{\right)}


%%%% Different kind of derivatives %%%%%

\newcommand{\delbar}{\bar{\partial}}
\newcommand{\pdu}{\frac{\partial}{\partial u}}
%\newcommand{\pd}[1][2]{\frac{\partial #1}{\partial #2}}

%%%%% Arrows %%%%%
%\renewcommand{\ra}{\rightarrow}                   % right arrow
\newcommand{\lra}{\longrightarrow}              % long right arrow
%\renewcommand{\la}{\leftarrow}                    % left arrow
\newcommand{\lla}{\longleftarrow}               % long left arrow
\newcommand{\ua}{\uparrow}                     % long up arrow
\newcommand{\na}{\nearrow}                      %  NE arrow
\newcommand{\llra}[1]{\stackrel{#1}{\lra}}      % labeled long right arrow
\newcommand{\llla}[1]{\stackrel{#1}{\lla}}      % labeled long left arrow
%\newcommand{\lua}[1]{\stackrel{#1}{\ua}}      % labeled  up arrow
%\newcommand{\lna}[1]{\stackrel{#1}{\na}}      % labeled long NE arrow
\newcommand{\xra}{\xrightarrow}
\newcommand{\into}{\hookrightarrow}
\newcommand{\tto}{\longmapsto}
\def\llra{\longleftrightarrow}

\def\d/{/\mspace{-6.0mu}/}
\newcommand{\git}[3]{#1\d/_{\mspace{-4.0mu}#2}#3}
\newcommand\zetahilb{\zeta_{{\text{Hilb}}}}
\def\Fy{\sF \mspace{-3.0mu} \cdot \mspace{-3.0mu} y}
\def\tv{\tilde{v}}
\def\tw{\tilde{w}}
\def\wt{\widetilde}
\def\wtilde{\widetilde}
\def\what{\widehat}

%%%%%%%%%%%%%%%%%%% Mark's definitions %%%%%%%%%%%%%%%%%%%%

\newcommand{\frakg}{\mbox{\frakturfont g}}
\newcommand{\frakk}{\mbox{\frakturfont k}}
\newcommand{\frakp}{\mbox{\frakturfont p}}
\newcommand{\q}{\mbox{\frakturfont q}}
\newcommand{\frakn}{\mbox{\frakturfont n}}
\newcommand{\frakv}{\mbox{\frakturfont v}}
\newcommand{\fraku}{\mbox{\frakturfont u}}
\newcommand{\frakh}{\mbox{\frakturfont h}}
\newcommand{\frakm}{\mbox{\frakturfont m}}
\newcommand{\frakt}{\mbox{\frakturfont t}}
\newcommand{\G}{\Gamma}
\newcommand{\g}{\gamma}
\newcommand{\fraka}{\mbox{\frakturfont a}}
\newcommand{\db}{\bar{\partial}}
\newcommand{\dbs}{\bar{\partial}^*}
\newcommand{\p}{\partial}
\renewcommand{\k}{\textbf{k}}
\newcommand{\rH}{\widetilde{H}}
\newcommand{\cH}{H^\ast}
\newcommand{\ccH}{\check{H}^*}
\newcommand{\ext}{\Lambda_{\IZ}}
\newcommand{\rp}{\IR\IP}
\newcommand{\hz}{\IZ/2\IZ}
\newcommand{\tf}{\wt{f}}
\newcommand{\form}{\Omega}
\newcommand{\dr}{H_{DR}}
\newcommand{\cdr}{H_{c}}
\newcommand{\plane}{\IR^2}
\newcommand{\pplane}{\IR^2 - \{0\}}
\newcommand{\fU}{\mathfrak{U}}
\newcommand{\cU}{\mathcal{U}}
\newcommand{\calH}{\mathcal{H}}
\newcommand{\cV}{\mathcal{V}}
\newcommand{\cI}{\mathcal{I}}
\newcommand{\cJ}{\mathcal{J}}
\newcommand{\cF}{\mathcal{F}}
\newcommand{\sH}{\mathcal{H}_{cv}}

%% Temporary Definitions %%
\newcommand{\noi}{dx_1 \wedge \cdots \what{dx_i} \cdots \wedge dx_n}
\newcommand{\ctsw}{dx_1 \wedge \cdots \wedge dx_n}
\newcommand{\w}{\omega}
\newcommand{\cp}{\IC\IP}
\newcommand{\cf}{f^*}
\newcommand{\Up}{U^{\pm}_i}
\newcommand{\Vp}{V^{\pm}_i}
\newcommand{\ppm}{\phi^{\pm}_i}
\newcommand{\why}{\what{y}_i}
\newcommand{\whx}{\what{x}_i}
\newcommand{\whz}{\what{z}_i}
\newcommand{\Wp}{W^{\pm}_i}
\newcommand{\wb}{\overline}
\newcommand{\cl}{\mathrm{cl}}

\newcommand{\s}{\sigma}
\newcommand{\lb}{\lambda}
\newcommand{\hint}{\int_{\IH^n}}
\newcommand{\hbint}{\int_{\p \IH^n}}
\newcommand{\sint}{\int_{-\infty}^\infty}
\newcommand{\intr}{\mathrm{int}\,}


%%%%%%%%%%%%% new definitions for the positive mass paper %%%%%%%%%

\newcommand{\sperp}{{\scriptscriptstyle \perp}}
\newcommand{\tG}{G_n(\IR^{n + k})}

%%%%%%%%%%%%%%%%%%%%%%%

%%%%%%%%%%%%%%%%%%%%%%%%%%%%%%%%%%%%%%%%%%%%%




%
\begin{document}
%

\title{Notes on Characteristic Classes}
\author{Ziquan Yang}


\date{\today}

\maketitle
 
\tableofcontents
%\setcounter{secnumdepth}{1} 

\section{Introduction}
\section{Algebraic Topology}
\subsection{Grassmann Manifolds}
\subsubsection{Universal bundle}
We first explain the ideas in this section. Many proofs are pretty technical topological arguments and are delayed. 

Given a curve $M^1 \subseteq \IR^{k + 1}$, we may define a map $$t : M^1 \to S^k$$ that sends every point in $M$ to a unit tangent vector. Of course, to do it consistently we need an orientation of $M^1$. However, when an orientation is not available or we may disregard it, it is always possible to define a map by passing to projective space, i.e. $t : M^1 \to \IP^k$ that sends a point to its tangent space. 

Similarly, given a hypersurface $M^k \subseteq \IR^{k + 1}$, we may define a map $$ n : M^k \to \IP^k$$ that sends a point to the orthogonal complement of its tangent space. 

More generally we would like a map for $M^n \subseteq \IR^{n + k}$, so we need to space to parametrize the $n$-dimensional subspaces of $\IR^{n + k}$. Therefore we introduce Grassmann manifolds. The underlying set $G_n(\IR^{n + k})$ is the $n$-dimensional subspaces of $\IR^{n + k}$ and we topologize it as follows: Let $$V_n(\IR^{n + k}) \subseteq \IR^{n + k} \times \cdots \times \IR^{n + k}$$ be the subset parametrizing the $n$-tuples of vectors $(v_1, \cdots, v_n)$ that are linearly independent (such tuple is called a $n$-frame). Clearly $V_n$ is a Zariski open subset. Each $n$-tuple in $V_n$ determines a $n$-dimensional subspace in $\IR^{n + k}$. Therefore there is a canonical function 
$$ q : V_n(\IR^{n + k}) \to G_n (\IR^{n + k}) $$ 
This map gives $G_n (\IR^{n + k})$ a quotient topology. Alternatively, we could have used the set of orthonomal $n$-frames $V^0_{n}$. $V^0_{n}$ is clearly compact, and hence $G_n (\IR^{n + k})$ is compact. We will verify later that $G_n (\IR^{n + k})$, which now is a priori only a topological space, is indeed a topological manifold. 

In particular $\IP^n$ can now be viewed as a special case of Grassmann manifolds, and we may define a canonical vector bundle on $G_n(\IR^{n + k})$ in a completely analogous fashion: $$\gamma^n(\IR^{n + k}) = \{ (\text{$n$-plane in }\IR^{n + k}, \text{vectors in the plane}) \}$$ 

With Grassmann mainfolds we can define generalized Gauss maps $\wb{g} : M \to \tG$ by sending a point to its tangent space. However, Grassmann manifolds can absorb much more bundles. In some sense it behaves like a terminal object in a category. 

\begin{lemma}
\label{complemma}
For any $n$-plane bundle $\xi$ over a compact base space $B$ there exists a bundle map $\xi \to \gamma^n(\IR^{n + k})$ provided that $k$ is sufficiently large. 
\end{lemma}

In order to generalize to common manifolds that do not happen to be compact, we do two things: replace ``compact" be ``paracompact" and introduce infinite Grassmann manifolds. A topological space is paracompact if it is Hausdorff and for every open cover, there is a locally finite refinement. Every metric space is paracompact, a direct limit of compact spaces is compact, and nearly all familiar topological spaces are paracompact. We realize the infinite Grassmann manifolds as a direct limit of the sequence 
$$ G_n(\IR^n) \subset G_n(\IR^{n+1}) \subset G_n(\IR^{n + 1}) \subset \cdots $$
To make sense of ``$\subset$": Define $\IR^\infty = \oplus_{i = 1}^\infty \IR_i$ and view each $\IR^n$ as a subset $\oplus_{i = 1}^n \IR_i$. We denote the above infinite Grassmann manifold as $G_n(\IR^\infty)$. As a special case, we have infinite projective space $\IP^\infty = G_1(\IR^\infty)$ is the direct limit of $\IP^1 \subset \IP^2 \subset \cdots$. 

A canonical bundle $\gamma^n$ over $G_n$ can be constructed just as in the finite case, i.e. as a subset of $G_n \times \IR^\infty$ that consists of pairs of planes and vectors in the planes. It is topologized as a subset of the Catesian product. We will verify that it does satisfy the local triviality condition and is a $n$-plane bundle over $G_n$. Now Lemma~\ref{complemma} can be extended to the theorem:
\begin{theorem}
Any $\IR^n$-bundle $\xi$ over a paracompact base admits a bundle map $\xi \to \gamma^n$. Moreover, any two such bundle maps are homotopic. 
\end{theorem}
In particular, the homotopic bundle maps induce homotopic maps on base spaces. Therefore any $\IR^n$-bundle $\xi$ over a paracompact space $B$ determines a unique homotopy class of maps $\wb{f_\xi} : B \to G_n$. 

\subsubsection{Cell Structure}
In this section we use $D^p$ to denote the closed unit disk in $\IR^p$. 
Let us recall the definition of a CW-complex:
\begin{definition} A CW-complex is a Hausdorff space $K$, together with a partitiion of $K$ into disjoint subsets $\{ e_\alpha \}$, such that 
\begin{enumerate}
\item Each $e_\alpha$ is topologically an open cell of dimension $n(e)$. For each $e_\alpha$, there is a characteristic map $\varphi_\alpha : D^{n(\alpha)} \to K$ that carries $\intr D^{n(\alpha)}$ homeomorphically onto $e_\alpha$. 
\item Each point $x \in \cl(e_\alpha) - e_\alpha$ must lie in a cell $e_\beta$ of lower dimension. 
\item Each point of $K$ is contained in a finite subcomplex. 
\item $K$ is topologized as a direct limit of its finite subcomplexes. 
\end{enumerate}
\end{definition}
The last two conditions always hold if the complex is finite (i.e. has finitely many cells), so we only use them to deal with infinite complexes. 


\section{Differential Geometry}

\section{Algebraic Geometry}

\end{document}