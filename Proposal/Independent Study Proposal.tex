\documentclass[12pt]{article}
\usepackage[utf8]{inputenc}
\usepackage{amsmath}
\usepackage{amsfonts}
\usepackage{amssymb}
\author{Ziquan Yang}
\title{Independent Study Proposal}

\newcommand{\IC}{\mathbb{C}}
\newcommand{\IP}{\mathbb{P}}
\newcommand{\IR}{\mathbb{R}}
\begin{document}

\section{Description of independent study}
\subsection{Description of the subject}
Vector bundles arise naturally in topology and geometry and  characteristic classes are important tools used to study vector bundles. For example, we may compute the Euler characteristic of a manifold from the Euler class (or top Chern class) of the tangent bundle. The theory on normal bunles of submanifolds tells us, say, $\IP^2(\IR)$ cannot be embedded in $\IR^3$. To my knowledge, there are two main approaches to characteristic classes, one algebraic topological, the other differential geometric. In \cite{Milnor}, characteristic classes are constructed in terms of singular cohomology. In \cite{Bott}, another standard reference in the field, they are constructed in terms of De Rham cohomology. There are some differences between the two approaches. For example, singular cohomology can take coeffients in almost any ring. De Rham cohomology mostly takes coefficients in $\IR, \IC$, so there are no torsion groups. Another difference is that integration plays a significant role in De Rham cohomology and Poincare duality, in this setting, is a pairing of De Rham cohomology and compactly supported cohomology, as opposed to that of cohomology and homology.

\subsection{Motivation}
My first experience with characteristic classes comes from my research with Prof. Schoen. We used the top Chern class to compute the Euler characteristic of a smooth hypersuraface of bidegree $(3, d)$ in $\IP^2 \times \IP^1$, which gives important information on singular fibers on the hypersurface with respect to the projection $\IP^2 \times \IP^1 \to \IP^1$. I was amazed by the power of the tool. After all, all the information we have for these hypersurfaces is their intersection type with $\IP^2 \times \cdot$ and $\cdot \times \IP^1$. 

Another motivation comes from the study of algebraic geometry in general. As I was trying to read through Hartshorne, I found it is getting increasingly demanding of a good sense of vector bundles, whose analogues in algebraic geometry are sheaves of modules. 

\subsection{My Background}
I have some background in related areas. I have taken the algebraic topology sequence (611-612), commutative algebra (602), differential geometry (621) and have some knowledge of algebraic geometry. In the fall I will be taking Riemann surfaces (625). 
 
\subsection{What I intend to learn}
The main goal of the course is to go as far as possible in \cite{Milnor}. The specific topics I am trying to learn include Stiefel-Whitney classes, Thom isomorphism, Euler classes, Chern classes, Pontrjagin classes, as well as their connections to algebraic geometry. I shall learn how to compute the characteristic classes of some algebraic varieties. Since I have not worked with higher homotopy groups before, I think I may need to acquire some familiarity with them in order to understand some obstruction theory. I also plan to read some classic papers in the area, if time permits.  

Besides \cite{Milnor}, some other books that may become useful references are also listed. Other reference materials may come up through the course. I will do exercises in \cite{Milnor}, as well as any problems that Prof. Bryant would like me to think about. 

\section{Final product}
I will try to produce notes and possibly a term paper. 

\section{Scheduled Meetings and Work Expectations}
\section{Grade Basis}


\begin{thebibliography}{9}

\bibitem{Milnor}
  J. W. Milnor and J. D. Stasheff, \textit{Characteristic Classes}, Princeton Univ. Press, 1974.
  
\bibitem{Bott}
R. Bott and L. Tu, \textit{Algebraic Topology and Differential Forms}, Graduate Texts in Mathematics, Vol. 82, Springer, New York, 1995.

\bibitem{Spanier}
E. Spanier, \textit{Algebraic Topology}, McGraw-Hill, 1966

\bibitem{Hatcher}
A. Hatcher, \textit{Algebraic Topology}, Cambridge University Press, Cambridge, UK, 2002

\bibitem{Hatcher2}
A. Hatcher, \textit{Vector Bundles and K-theory}, unpublished online notes, availabel at \textit{http://www.math.cornell.edu/~hatcher/VBKT/VB.pdf}

\end{thebibliography}

\end{document}