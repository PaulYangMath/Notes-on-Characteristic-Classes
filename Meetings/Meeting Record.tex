\documentclass[12pt]{article}
\usepackage{amsmath}
\usepackage{amsthm}
\usepackage{amsfonts}
\usepackage{amssymb}
\usepackage{latexsym}
\usepackage{tikz} 
\usepackage{esint} 
%\usepackage{epsfig}
%\usepackage{graphicx}
%\usepackage[dvips]{graphicx}
\usepackage{tikz}
\usepackage{tikz-cd}




\usepackage[matrix,tips,graph,curve]{xy}

\newcommand{\mnote}[1]{${}^*$\marginpar{\footnotesize ${}^*$#1}}
\linespread{1.065}

\makeatletter

\setlength\@tempdima  {5.5in}
\addtolength\@tempdima {-\textwidth}
%\addtolength\hoffset{-0.5\@tempdima}
\addtolength\hoffset{-0.6\@tempdima}
\setlength{\textwidth}{5.5in}
\setlength{\textheight}{8.75in}
%\addtolength\voffset{-0.625in}
\addtolength\voffset{-1.2in}

\makeatother

\makeatletter 
\@addtoreset{equation}{section}
\makeatother


\renewcommand{\theequation}{\thesection.\arabic{equation}}

\theoremstyle{plain}
\newtheorem{theorem}[equation]{Theorem}
\newtheorem{corollary}[equation]{Corollary}
\newtheorem{lemma}[equation]{Lemma}
\newtheorem{proposition}[equation]{Proposition}
\newtheorem{conjecture}[equation]{Conjecture}
\newtheorem{fact}[equation]{Fact}
\newtheorem{facts}[equation]{Facts}
\newtheorem*{theoremA}{Theorem A}
\newtheorem*{theoremB}{Theorem B}
\newtheorem*{theoremC}{Theorem C}
\newtheorem*{theoremD}{Theorem D}
\newtheorem*{theoremE}{Theorem E}
\newtheorem*{theoremF}{Theorem F}
\newtheorem*{theoremG}{Theorem G}
\newtheorem*{theoremH}{Theorem H}

\theoremstyle{definition}
\newtheorem{definition}[equation]{Definition}
\newtheorem{definitions}[equation]{Definitions}
%\theoremstyle{remark}

\newtheorem{remark}[equation]{Remark}
\newtheorem{remarks}[equation]{Remarks}
\newtheorem{exercise}[equation]{Exercise}
\newtheorem{example}[equation]{Example}
\newtheorem{examples}[equation]{Examples}
\newtheorem{notation}[equation]{Notation}
\newtheorem{question}[equation]{Question}
\newtheorem{assumption}[equation]{Assumption}
\newtheorem*{claim}{Claim}
\newtheorem{answer}[equation]{Answer}
%%%%%% letters %%%%

\newcommand{\IA}{\mathbb{A}}
\newcommand{\IB}{\mathbb{B}}
\newcommand{\IC}{\mathbb{C}}
\newcommand{\ID}{\mathbb{D}}
\newcommand{\IE}{\mathbb{E}}
\newcommand{\IF}{\mathbb{F}}
\newcommand{\IG}{\mathbb{G}}
\newcommand{\IH}{\mathbb{H}}
\newcommand{\II}{\mathbb{I}}
\newcommand{\IK}{\mathbb{K}}
\newcommand{\IL}{\mathbb{L}}
\newcommand{\IM}{\mathbb{M}}
\newcommand{\IN}{\mathbb{N}}
\newcommand{\IO}{\mathbb{O}}
\newcommand{\IP}{\mathbb{P}}
\newcommand{\IQ}{\mathbb{Q}}
\newcommand{\IR}{\mathbb{R}}
\newcommand{\IS}{\mathbb{S}}
\newcommand{\IT}{\mathbb{T}}
\newcommand{\IU}{\mathbb{U}}
\newcommand{\IV}{\mathbb{V}}
\newcommand{\IW}{\mathbb{W}}
\newcommand{\IX}{\mathbb{X}}
\newcommand{\IY}{\mathbb{Y}}
\newcommand{\IZ}{\mathbb{Z}}

%%%%%%% macros %%%%%

%% my definitions %%%

\newcommand{\End}{\mathrm{End}}
\newcommand{\tr}{\mathrm{tr}}
%\newcommand{\ind}{\mathrm{ind}}

\renewcommand{\index}{\mathrm{index \,}}
\newcommand{\Hom}{\mathrm{Hom}}
\newcommand{\Aut}{\mathrm{Aut}}
\newcommand{\Trace}{\mathrm{Trace}\,}
\newcommand{\Res}{\mathrm{Res}\,}
%\newcommand{\rank}{\mathrm{rank}}
%\renewcommand{\dim}{\mathrm{dim}}

\renewcommand{\deg}{\mathrm{deg}}
\newcommand{\spin}{\rm Spin}
\newcommand{\supp}{\mathrm{supp \,}}
\newcommand{\Spin}{\rm Spin}
\newcommand{\erfc}{\rm erfc\,}
\newcommand{\sgn}{\rm sgn\,}
\newcommand{\Spec}{\rm Spec\,}
\newcommand{\diag}{\rm diag\,}
\newcommand{\Fix}{\mathrm{Fix}}
\newcommand{\Ker}{\mathrm{Ker \,}}
\newcommand{\Coker}{\mathrm{Coker \,}}
\newcommand{\Sym}{\mathrm{Sym \,}}
\newcommand{\Hess}{\mathrm{Hess \,}}
\newcommand{\grad}{\mathrm{grad \,}}
\newcommand{\Center}{\mathrm{Center}}
\newcommand{\Lie}{\mathrm{Lie}}
\newcommand{\coker}{\mathrm{coker}}

\newcommand{\ch}{\rm ch} % Chern Character

\newcommand{\rank}{\rm rank} 
%\renewcommand{\c}{\rm c}  % Chern class

\newcommand\QED{\hfill $\Box$} %{\bf QED}} 

\newcommand\Pf{\nonintend{\em Proof. }}


\newcommand\reals{{\mathbb R}} 
\newcommand\complexes{{\mathbb C}}
\renewcommand\i{\sqrt{-1}}
\renewcommand\Re{\mathrm Re}
\renewcommand\Im{\mathrm Im}
\newcommand\integers{{\mathbb Z}}
\newcommand\quaternions{{\mathbb H}}


\newcommand\iso{\,{\cong}\,} 
\newcommand\tensor{{\otimes}}
\newcommand\Tensor{{\bigotimes}} 
\newcommand\union{\bigcup} 
\newcommand\onehalf{\frac{1}{2}}
%\newcommand\Sym[1]{{Sym^{#1}(\complexes^2)}}

\newcommand\lie[1]{{\mathfrak #1}} 
\newcommand\smooth{\mathcal{C}^{\infty}}
\newcommand\trivial{{\mathbb I}}
\newcommand\widebar{\overline}

%%%%%Delimiters%%%%

\newcommand{\<}{\langle}
\renewcommand{\>}{\rangle}

%\renewcommand{\(}{\left(}
%\renewcommand{\)}{\right)}


%%%% Different kind of derivatives %%%%%

\newcommand{\delbar}{\bar{\partial}}
\newcommand{\pdu}{\frac{\partial}{\partial u}}
%\newcommand{\pd}[1][2]{\frac{\partial #1}{\partial #2}}

%%%%% Arrows %%%%%
%\renewcommand{\ra}{\rightarrow}                   % right arrow
\newcommand{\lra}{\longrightarrow}              % long right arrow
%\renewcommand{\la}{\leftarrow}                    % left arrow
\newcommand{\lla}{\longleftarrow}               % long left arrow
\newcommand{\ua}{\uparrow}                     % long up arrow
\newcommand{\na}{\nearrow}                      %  NE arrow
\newcommand{\llra}[1]{\stackrel{#1}{\lra}}      % labeled long right arrow
\newcommand{\llla}[1]{\stackrel{#1}{\lla}}      % labeled long left arrow
%\newcommand{\lua}[1]{\stackrel{#1}{\ua}}      % labeled  up arrow
%\newcommand{\lna}[1]{\stackrel{#1}{\na}}      % labeled long NE arrow
\newcommand{\xra}{\xrightarrow}
\newcommand{\into}{\hookrightarrow}
\newcommand{\tto}{\longmapsto}
\def\llra{\longleftrightarrow}

\def\d/{/\mspace{-6.0mu}/}
\newcommand{\git}[3]{#1\d/_{\mspace{-4.0mu}#2}#3}
\newcommand\zetahilb{\zeta_{{\text{Hilb}}}}
\def\Fy{\sF \mspace{-3.0mu} \cdot \mspace{-3.0mu} y}
\def\tv{\tilde{v}}
\def\tw{\tilde{w}}
\def\wt{\widetilde}
\def\wtilde{\widetilde}
\def\what{\widehat}

%%%%%%%%%%%%%%%%%%% Mark's definitions %%%%%%%%%%%%%%%%%%%%

\newcommand{\frakg}{\mbox{\frakturfont g}}
\newcommand{\frakk}{\mbox{\frakturfont k}}
\newcommand{\frakp}{\mbox{\frakturfont p}}
\newcommand{\q}{\mbox{\frakturfont q}}
\newcommand{\frakn}{\mbox{\frakturfont n}}
\newcommand{\frakv}{\mbox{\frakturfont v}}
\newcommand{\fraku}{\mbox{\frakturfont u}}
\newcommand{\frakh}{\mbox{\frakturfont h}}
\newcommand{\frakm}{\mbox{\frakturfont m}}
\newcommand{\frakt}{\mbox{\frakturfont t}}
\newcommand{\G}{\Gamma}
\newcommand{\g}{\gamma}
\newcommand{\fraka}{\mbox{\frakturfont a}}
\newcommand{\db}{\bar{\partial}}
\newcommand{\dbs}{\bar{\partial}^*}
\newcommand{\p}{\partial}
\renewcommand{\k}{\textbf{k}}
\newcommand{\rH}{\widetilde{H}}
\newcommand{\cH}{H^\ast}
\newcommand{\ccH}{\check{H}^*}
\newcommand{\ext}{\Lambda_{\IZ}}
\newcommand{\rp}{\IR\IP}
\newcommand{\hz}{\IZ/2\IZ}
\newcommand{\tf}{\wt{f}}
\newcommand{\form}{\Omega}
\newcommand{\dr}{H_{DR}}
\newcommand{\cdr}{H_{c}}
\newcommand{\plane}{\IR^2}
\newcommand{\pplane}{\IR^2 - \{0\}}
\newcommand{\fU}{\mathfrak{U}}
\newcommand{\cU}{\mathcal{U}}
\newcommand{\calH}{\mathcal{H}}
\newcommand{\cV}{\mathcal{V}}
\newcommand{\cI}{\mathcal{I}}
\newcommand{\cJ}{\mathcal{J}}
\newcommand{\cF}{\mathcal{F}}
\newcommand{\sH}{\mathcal{H}_{cv}}

%% Temporary Definitions %%
\newcommand{\noi}{dx_1 \wedge \cdots \what{dx_i} \cdots \wedge dx_n}
\newcommand{\ctsw}{dx_1 \wedge \cdots \wedge dx_n}
\newcommand{\w}{\omega}
\newcommand{\cp}{\IC\IP}
\newcommand{\cf}{f^*}
\newcommand{\Up}{U^{\pm}_i}
\newcommand{\Vp}{V^{\pm}_i}
\newcommand{\ppm}{\phi^{\pm}_i}
\newcommand{\why}{\what{y}_i}
\newcommand{\whx}{\what{x}_i}
\newcommand{\whz}{\what{z}_i}
\newcommand{\Wp}{W^{\pm}_i}
\newcommand{\wb}{\overline}
\newcommand{\cl}{\mathrm{cl}}

\newcommand{\s}{\sigma}
\newcommand{\lb}{\lambda}
\newcommand{\hint}{\int_{\IH^n}}
\newcommand{\hbint}{\int_{\p \IH^n}}
\newcommand{\sint}{\int_{-\infty}^\infty}
\newcommand{\intr}{\mathrm{int}\,}

\newcommand{\st}{\mathrm{s.t.}\,}
%%%%%%%%%%%%% new definitions for the positive mass paper %%%%%%%%%

\newcommand{\sperp}{{\scriptscriptstyle \perp}}
\newcommand{\tG}{G_n(\IR^{n + k})}
\newcommand{\Ext}{\mathrm{Ext}}
\newcommand{\proj}{\mathrm{proj}}
\newcommand{\invlim}{\varprojlim}

\newcommand{\Exp}{\mathrm{Exp}}
\newcommand{\sm}{\varepsilon}
\newcommand{\Ohm}{\Omega}
\newcommand{\Sq}{\mathrm{Sq}}
\newcommand{\id}{\mathrm{id}}
\newcommand{\fN}{\mathfrak{N}}
\newcommand{\fm}{\mathfrak{m}}
%%%%%%%%%%%%%%%%%%%%%%%

%%%%%%%%%%%%%%%%%%%%%%%%%%%%%%%%%%%%%%%%%%%%%





\begin{document}


\title{Meeting Record}
\author{Ziquan Yang}


\date{\today}

\maketitle
 
\tableofcontents
%\setcounter{secnumdepth}{1} 

\section{Meeting on Sept. 2nd}
\begin{enumerate}
\item Prof: What do $\IP^n$'s look like, for small $n$'s? \\
Me: (I drew the standard planar representation of $\IR IP^2$ that identifies the boundary of a square using antipodal maps.) $\IP^1$ is topologically the same as $\IS^1$. 

\item Prof: Is $\IP^2$ orientable? \\
Me: No. Ah, here is a quick proof. If $\IP^2$ is orientable, then all open submanifolds of it are, but the Mobius strip is an open submanifold of $\IP^2$. 

\item Prof: How about $\IP^3$? Is it orientable?
Me: My impression is that it is not. (Then I spent a while trying to come up with a quick proof)

\item Me: So I am going to cheat and use the fact that a manifold $M$ is orientable if and only if $w_1(M) = 0$. 

\item Prof: All line bundle over paracompact spaces are isomorphic to their duals. \\
Me: Emm, I can check that myself later. \\
Prof: The space of inner products on $\IR$ has a canonical orientation. Well, what is the dimension of the space of inner products of $\IR^n$? \\
Me: They correspond to $n \times n$ matrices, so I guess it is $n^2$?\\
Prof: But they have to be symmetric. \\
Me: Right. So the answer is $n(n+1)/2$. \\
Prof: Then what is the dimension (of the inner product space) when $n = 1$?
Me: 1. 

\item Prof: We can give a natural orientation on the inner product space by choosing the positive ones, i.e. those that send $(v, v)$ to a positive number, to be a basis. You can then use partition of unity to construct a nowhere vanishing section. 

\item Prof: What are other common manifolds of dimension $3$ do you know? \\
Me: Not really. \\
Prof: How about the space $O(n) = \{ M \in \IR^{n \times n} : M^T M = 1 \}$? How many equations are used to describe the condition? \\
Me: 9, but (trying to count how many equations are repeated)...\\
Prof: $M^T M$ is symmetric, so...\\
Me: Ah, so there are actually only $6$ equations. (thinking for some seconds) The other equations are algebraically independent, since they each introduce a new variable. Therefore we cannot further reduce the number of equations. 

\item Prof: Yes, but it may happen that the intersection is nasty somewhere. How do you want to show that it is smooth? \\
Me: Write down the equations using coordinates and then compute the Jacobian. Show that the Jacobian has appropriate rank. \\
Prof: Right, what is the theorem that you are using? \\
Me: If $f : \IR^m \to \IR^n$ is a smooth map and $p \in \IR^n$ is a regular value, then $f^{-1}(p)$ is a smooth submanifold of $\IR^m$. \\
Prof: Right, what is the name of the theorem?\\
Me: Regular value theorem? Or something like that? \\
Prof: It's the implicit function theorem. 

\item Me: My intuition is that $O(3)$ is homogeneous, so it suffices to check one point, for which we can simply use $I$. 

\item Prof: That seems to be a lot of computations. Could you show that the derivative is surjective directly? So how would you compute the derivative? \\
Me: The derivative of $M \to M^T T$ is 
$$ \lim_{\| M \| \to 0} \frac{(I + M)^T(I + M) - I}{\| M \|} $$
Prof: For a function $f : \IR^n \to \IR^m$, you should be thinking of this formula:
$$ f'(x)(v) = \lim_{t \to 0} \frac{f(x + tv) - f(x)}{t} $$ 
Me: But that computes the image of one $v$ at a time. I guess if you let $v$ run over a basis, you can reconstruct the derivative $f'(x)$. \\

\item Me: So using this formula, the answer is 
$$ \lim_{t \to 0} \frac{(I  + tM)^T(I + tM) - I}{t} = \lim_{t \to \infty} \frac{I + tM + tM^T + t^2 M^T M - I}{t} = M + M^T $$
Prof: So is this map surjective? \\
Me: So that is essentially asking whether a symmetric $3 \times 3$ matrix $A$ can always be written in the form $M + M^T$. We can show this by doing a dimension counting argument. If $A = [ a_1, a_2, \cdots, a_9]$, then we can let $M = \cdots $\\
Prof: Don't use coordinates. Can you do this directly? \\
Me: Ah, so probably we can put $M = (1/2)(A + \cdot)$. (I wanted to put $(A + A^T)/2$, but something is weird since $A = A^T$)\\
Prof: You are done. \\
Me: Haha, yes, $M = A/2$ works.  

\item Me: So what does $O(3)$ look like? \\
Prof: That is an interesting question. Is it connected?
Me: No. Those matrices with positive determinants and those with negative determinants should be on different components. 
Prof: Yes, there are actually only two components. Those with positive determinants form a group $SO(3)$. Is it compact?\\
Me: Emm, it is not clear from the defining equations. To show it is compact we need it to be closed and bounded. It is clearly closed. So now we only need to show it is bounded. Why don't use simply use a ``stupid" norm? If $M = [a_{ij}]$, then define $\| M \| = \max \{a_{ij} \}$. The intuition is that $M$ cannot be too big with respect to this norm. Otherwise it is impossible that $M^T M = I$. \\
Me: $\| M^T M \| \ge \| M \|^2$. This norm induces the same topology as the norm that we normally use. So $O(3)$ is bounded. \\
Me: Hmm, so can it happen to be some other common manifolds? It is probably not the same as $\IP^3$. \\
Prof: It turns out that it is homeomorphic to $\IP^3$, although it is by no means obvious so far. 

\item Prof: Do you know the ``bell trick" that can readily show that $\pi_1(O(3))$? \\
Me: That may be similar to the argument that we used to show that $\pi_1(\IP^2) = \IZ/2\IZ$. 
\end{enumerate}



\begin{thebibliography}{9}


\bibitem{Milnor}
  J. Milnor, J. Stasheff, \textit{Characteristic classes}, Annals of mathematics studies, Princeton university press, Number 76, 1978
\end{thebibliography}



\end{document}
